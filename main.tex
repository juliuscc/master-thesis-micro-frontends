\documentclass{report}
\usepackage[utf8]{inputenc}

\title{Migrating to Micro Frontends -- Taking it One Step at a Time}
\author{Julius Celik \texttt{<jcelik@kth.se>}}
\date{\today}

\usepackage[euler]{textgreek}

\newcommand{\fe}{\textmugreek FE}

\setlength{\parindent}{0em}
\setlength{\parskip}{1em}

\usepackage{biblatex}
\addbibresource{references.bib}

% For acronyms in the text
\usepackage[printonlyused]{acronym}

% For images
\usepackage{graphicx}

% For blockquotes
\usepackage{csquotes}


\begin{document}

\maketitle

\begin{acronym}[RDBMS]
\acro{BE}{Back End}
\acro{DX}{Developer Experience}
\acro{FE}{Front End}
\acro{MFE}{Micro Frontend}
\acro{UX}{User Experience}
\end{acronym}
% 
\chapter{Introduction}

\section{Background}
\acp{MFE} is a web \ac{FE} architecture style, where the \ac{FE} is composed of multiple simultaneously running applications. As the name implies it is heavily inspired by the microservice architecture style. It promises \ac{DX} improvements similar to those that microservices provide, like team independence, clear code responsibility, and high cohesion. Additionally \acp{MFE} promises a higher cohesion between \ac{BE} and \ac{FE}, mitigating feature latency that comes from asynchronous deployment. In practice this means that the delay from starting to implement a feature to it reaching market, should be decreased when using \acp{MFE} \cite{Celik}. \textbf{ADD SOURCES!!!}


\section{Problem Statement}

Write about how teams should be very divided and independent

Write about how teams would like to try \ac{MFE} but should not edit code from other teams.

\section{Problem}

Is it possible to implement \ac{MFE} without changing other parts of the code base? Location transparency and self contained micro frontends. Three artifacts are introduced:

\begin{description}
\item[Construct] Self contained micro frontends.
\item[Instantiation] Plutt -- A build tool for self contained micro frontends.
\item[Method] The method of creating self contained micro frontends.
\end{description}


\textbf{Old problem}

The feasibility of \acp{MFE} are unknown, as the performance impacts are not known. It is believed that \ac{DX} could be improved by using \acp{MFE}, but the technology is criticised for possibly introducing a large negative performance impact, which in turn affects \ac{UX}.


It would be interesting to measure the \ac{UX} performance impact when using \acp{MFE}. If it could be shown that a complex production facing web application could be transformed, to use a \ac{MFE} architecture \textit{without} a significant performance impact, the criticism regarding performance can be rejected. The problem can be condensed into the following research question:

\begin{center}
\begin{tabular}{p{0.28\textwidth}p{0.60\textwidth}}
     \textbf{Research Question:} & Is it possible to transform a monolithic complex production facing web application, to use a \ac{MFE} architecture, without \ac{UX} being significantly affected?
\end{tabular}
\end{center}

\section{Hypothesis}
It is possible to implement a monolithic complex web application using a \ac{MFE} architecture, without a large negative impact on performance and \ac{UX}.

\section{Purpose}
The purpose is to evaluate the feasibility of using \acp{MFE}, regarding \ac{UX}. The impact on \ac{DX} will not be evaluated. Assuming that \ac{DX} is positively impacted, there is a value in knowing if there is a trade-off between \ac{DX} and \ac{UX} or if \ac{UX} impacts are negligible.

\section{Goal}
The goal is to try to transform a monolithic complex web app, to using a \ac{MFE} architecture and compare performance between the monolithic web page and the \ac{MFE} web page. The attempt is that the \ac{MFE} page will have similar performance to the monolithic page, as this would mean that only \ac{DX} has to be considered when evaluating the use of \acp{MFE}.

\section{Tasks}
Initially a literature study will be conducted with two purposes:

\begin{enumerate}
    \item Evaluate the different methods for implementing a \ac{MFE} architecture.
    \item Decide on quantifiable performance metrics that have shown an impact on \ac{UX}. Likely metrics could be bundle size, time to first interaction, or time to first render. As there exists extensive research into this field, the chosen metrics will be a very trustworthy measurement of \ac{UX} impact.
\end{enumerate}

When an implementation and evaluation method is chosen, a complex monolithic web application will be re-implemented, using a \ac{MFE} architecture. When the modified web page is created the original and modified page can be compared, using the chosen performance metrics.

Finally the results will be evaluated and analyzed. The research question will be answered. All of this will be compiled into the thesis.

\section{Research Methodology}

\section{Old Method}
The project will use an empirical method. There exists quantifiable measurements, and the correlation between these and \ac{UX} are very extensively proven. Therefore, there exists a good foundation for conducting performance tests on web pages, and then from analysis of these measurements deduce \ac{UX} impact.

The biggest flaw with the chosen method is that it does not provide any method for proving that \acp{MFE} lead to a negative \ac{UX} impact. If the modified page is significantly worse than the original page it only proves that this specific page became worse, which could be because of a poor implementation. If the modified page has similar performance as the original page it proves that it is possible to create a \ac{MFE} web page with no significant performance degradation. Therefore, the hypothesis can be proven, but not rejected.

\section{Supervisor and Examiner}

My supervisor from KTH will be Martin Monperrus. My examiner from KTH will be Benoit Baudry.

The project will be conducted at DigitalRoute who will provide Tommy Gunnarsson as a supervisor. They will also provide me with any necessary equipment like a computer, and access to all of the development tools used at DigitalRoute.

\section{Eligibility and study planning}

All my courses from my bachelor are completed. I have completed more than 60 credits of advanced courses in my master. During my master thesis, I will conduct one 7.5 credit course, and after my master thesis I will have finished all courses for my master.

\section{Milestone chart}

The project will start on 13 January and end on 29 May. There will be the following milestones:

\textbf{7 February:} The project plan and literature study is finished. At this stage the method for implementation and performance evaluation will be chosen.

\textbf{17 April:} The modified web page will be finished. At this stage performance tests can be conducted.

\textbf{15 May:} The first draft of the Thesis will be finished. If it is accepted a thesis presentation date can be chosen.

\textbf{22 May:} The project presentation has been conducted and peer reviews have been provided.

\textbf{29 May:} The final thesis report is submitted.

\chapter{Background}

\section{Micro Frontends}
Micro frontends is an architectural style for scaling \ac{FE} development, so that many teams can work on the same project by enabling team independence \cite{Jackson2019}. It relies on dividing an \ac{FE} into multiple independently deliverable components that can be composed into a cohesive \ac{FE}. \acp{MFE} are related to microservices and share many of the same characteristics \cite[ch.~1]{Gears2020}. An important common aspect is the possibility for independent deployability \cite{Jackson2019}, where teams can deploy any changes to software owned by them, without affecting other teams. Some of the benefits from using \acp{MFE} are: simple decoupled codebases, independent deployment, autonomous teams \cite{Jackson2019}, and greater customer focus \cite[ch.~1]{Gears2020}.

According to some \acp{MFE} are related to vertical slicing \cite[ch.~1]{Gears2020}, a software decomposition strategy based on composing software in functionally coherent slices that fully implement features \cite{Ratner2011}. This is opposite to horizontal slicing, where software is composed of modules that fulfill the same technological purpose like database-, entity-, controller-, boundary-layer \cite{Ratner2011}.

There are many different integration approaches, for bundling the final \ac{FE} into a combined entity. The main categories are build-time integration, server-side integration, and client-side integration \cite{Jackson2019, Gears2020}. There is also route based integration or loosely coupled \acp{MFE} where the \ac{FE} consists of separate web pages served on different routes, that are connected with links \cites[ch.~2]{Gears2020}{ Yang2019}\cite{Yang2019}. Build-time integration does not fit all definitions of \acp{MFE} \cite{Gears2020} and there exists a consensus that using build-time integration looses many of the benefits of using \acp{MFE} \cite{Jackson2019}.

Server-side integration is when a \ac{FE} is split into fragments, which are combined in run-time \cite{Jackson2019, Gears2020}. This can be done by using Server Side Includes \cite{Gears2020,Jackson2019,Fagan}. A drawback with using server-side integration is that it does not enable dynamic client interactions. \cite[ch.~4]{Gears2020}. Dynamic in this case refers to the client application being able to update based on user interactions, without a page reload.

Client-side integration techniques are the most discussed \acp{MFE} techniques, as many modern web pages require dynamic application design. Client-side integration techniques can be categorised into many categories like: iframes, web components, and meta frameworks (also know as unified single page applications). \textbf{cite at least jackson and gears. Maybe also the google scholar one}

\section{Client-Side Integration Techniques}

% \section{\ac{MFE} frameworks}


\chapter{Related work}


\section{Micro Frontends}
Micro frontends is a front end technique that originates from microservices \cite{Jackson2019}. Where microservices aim to solve scalability problems problems in the back-end, micro frontends aim to solve the same problems in the front-end, by applying many of the same concepts and methods. There does not exist one single definition but one of the introducers of micro frontends, ThoughtWorks, define microfrontends as:
\blockquote{An architectural style where independently deliverable frontend applications are composed into a greater whole \cite{Jackson2019}}

An important common aspect is the possibility for independent deployability \cite{Jackson2019}, where teams can deploy any changes to software owned by them, without affecting other teams. The way this is achieved is by using vertical slicing \cite[ch.~1]{Geers2020}, a software decomposition strategy based on composing software in functionally coherent slices that fully implement features \cite{Ratner2011}. This is opposite to horizontal slicing, where software is composed of modules that fulfill the same technological purpose like database-, entity-, controller-, boundary-layer \cite{Ratner2011}. The evolution of decomposition strategies are presented in Figure \ref{fig:vertical-slicing}. Some of the promised benefits from using micro frontends are: simple decoupled codebases, independent deployment, autonomous teams \cite{Jackson2019}, and greater customer focus \cite[ch.~1]{Geers2020}. 

There are many different integration approaches for bundling the final front end into a combined entity. The main categories are: 1) build-time integration, 2) server-side integration, and 3) client-side integration \cite{Jackson2019, Geers2020}. There is also route based integration or loosely coupled micro frontends where the front end consists of separate web pages served on different routes, that are connected using only hyperlinks \cites[ch.~2]{Geers2020}{Yang2019}. Build-time integration does not fit all definitions of micro frontends \cite{Geers2020} and there exists a consensus that using build-time integration looses many of the benefits of using micro frontends \cite{Jackson2019}.

Server-side integration is when a front end is split into fragments, which are combined in run-time \cite{Jackson2019, Geers2020}. This can be done by using Server Side Includes \cite{Geers2020,Jackson2019,Fagan}. A drawback with using server-side integration is that it does not enable dynamic client interactions. \cite[ch.~4]{Geers2020}. Dynamic in this case refers to the client application being able to update the graphical user interface, based on user interactions, without a page reload.

Client-side integration techniques are the most discussed micro frontends techniques, as many modern web pages require dynamic application design. Client-side integration techniques can be categorised into categories like: iframes \cites{Jackson2019}[ch.~2]{Geers2020}, web components \cites{Jackson2019}[ch.~5]{Geers2020}, and meta frameworks (also know as unified single page applications) \cite[ch.~6]{Geers2020}.


\section{Client-Side Integration Techniques}

The most naive integration method is using iframes which is a web native standard \cites{Jackson2019}[ch.~2]{Geers2020}. Using iframes are in essence using a micro frontend architecture, and they were officially introduced with the HTML 4.01 standard in 1998 \cite{Raggett1999}. Iframes comes with some negative trade-offs like performance overhead, accessibility problems, search engine optimization problems, and the lack of layout control \cite[ch.~2]{Geers2020}.

A more modern alternative to iframes are web components that is a new collection of web native standards \cite{MDNWebDocs}. They include APIs that allow custom elements to be defined and shadow DOM which is an API for encapsulating elements \cite{MDNWebDocs}. Shadow DOM allows styling and other properties to be scoped to a specific part of a web page, to limit interference between different components \cite{WebComponents.org}. It is not necessary to use custom elements together with shadow DOM, and the appropriate combination can be applied depending on the use case \cite{Geers2020}.

Web components are a low level construct, and therefore many frameworks exist that uses web components as a compilation target. Polymer is a large framework suite, that can be used to create web components \cite{ThePolymerProject,ThePolymerProjecta}. Polymer is developed by Google \cite{ThePolymerProjectb} and used in Google products like Youtube and Google Earth \cite{ThePolymerProjectc}. SkateJS \cite{SkateJS}, Slim.js \cite{Slim.js,Slim.jsa}, hybrids \cite{Lubanski}, and snuggsi \cite{DevPunks} are other light weight web component frameworks. Stensil is a web component tool set that allows applications written in the traditional frameworks Angular, React, Vue, Ember to become web components \cite{StencilJS}. Angular \cite{Googlec} and Vue \cite{Vuejs} has native support for creating web components.

Web components are supported in all major browsers: Edge, Firefox, Safari, Chrome and Opera \cite{WebComponents.orga}. There are fewer drawback with using web components compared with using iframes, but some still exist. The largest issue is that web components don't support progressive enhancement, which is when the front end is prerendered on the server before being served \cite[ch.~5]{Geers2020}. This can become a problem for performance, accessibility and search engine optimization. Another drawback, compared with using compilation-time integration is the lack of static annotations and analysis. Web components can be internally statically analysed but there does not exist any mechanism to share static annotations across the boundary of a web component.

There exists micro frontend frameworks that do not utilize web components like react-async-component \cite{Matheson}. One of the most popular and extensive frameworks is single-spa \cite{Single-spa}. It is a shell application that includes other application written in other frameworks \cite{Single-spa}. This kind of framework is called a unified single-page app, as it wraps other single-page-applications into one cohesive application \cite{Geers2020}. Single-spa requires that the root application be written as a single-spa application, which implies a large migration cost to existing applications \cite{Single-spa}. No micro frontend frameworks provide static analysis between the application boundaries.


% \include{sections/4-method}


\chapter{Misc}

\section{How to load the JS}
Write about the decision of how to load/fetch JS.

\begin{itemize}
    \item Append script to page. 
    
    I got the idea from this: (https://www.danielcrabtree.com/blog/25/gotchas-with-dynamically-adding-script-tags-to-html)
    
    I used this: (https://stackoverflow.com/a/27468484)
    \item Dynamic import. "Dynamic import is useful in situations where you wish to load a module conditionally, or on-demand." \cite{MDNWebDocs2020}
\end{itemize}

\section{Webpack vs Rollup}

\printbibliography

\end{document}
