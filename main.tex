\documentclass{report}
% \usepackage[utf8]{inputenc}

\title{Plutt: Strict Type Checking, Semantic Versioning, and Access Transparency in Micro Front-end Components}
\author{Julius Celik \texttt{<jcelik@kth.se>}}
\date{\today}

\usepackage[euler]{textgreek}

\setlength{\parindent}{0em}
\setlength{\parskip}{1em}

\usepackage{biblatex}
\addbibresource{references.bib}

% For acronyms in the text
\usepackage[printonlyused]{acronym}

% For images
\usepackage{graphicx}

% For blockquotes
\usepackage{csquotes}

\usepackage{enumitem}

% For subfigures
\usepackage{subcaption}

% For code blocks
\usepackage{listings}

% For custom enumeration
\usepackage{enumitem}

\begin{document}

\maketitle

\setlength{\parskip}{0em}
\tableofcontents
\setlength{\parskip}{1em}


\begin{acronym}[RDBMS]
\acro{BE}{Back End}
\acro{DX}{Developer Experience}
\acro{FE}{Front End}
\acro{MFE}{Micro Frontend}
\acro{UX}{User Experience}
\end{acronym}
% 
\chapter{Introduction}

\section{Background}
\acp{MFE} is a web \ac{FE} architecture style, where the \ac{FE} is composed of multiple simultaneously running applications. As the name implies it is heavily inspired by the microservice architecture style. It promises \ac{DX} improvements similar to those that microservices provide, like team independence, clear code responsibility, and high cohesion. Additionally \acp{MFE} promises a higher cohesion between \ac{BE} and \ac{FE}, mitigating feature latency that comes from asynchronous deployment. In practice this means that the delay from starting to implement a feature to it reaching market, should be decreased when using \acp{MFE} \cite{Celik}. \textbf{ADD SOURCES!!!}


\section{Problem Statement}

Write about how teams should be very divided and independent

Write about how teams would like to try \ac{MFE} but should not edit code from other teams.

\section{Problem}

Is it possible to implement \ac{MFE} without changing other parts of the code base? Location transparency and self contained micro frontends. Three artifacts are introduced:

\begin{description}
\item[Construct] Self contained micro frontends.
\item[Instantiation] Plutt -- A build tool for self contained micro frontends.
\item[Method] The method of creating self contained micro frontends.
\end{description}


\textbf{Old problem}

The feasibility of \acp{MFE} are unknown, as the performance impacts are not known. It is believed that \ac{DX} could be improved by using \acp{MFE}, but the technology is criticised for possibly introducing a large negative performance impact, which in turn affects \ac{UX}.


It would be interesting to measure the \ac{UX} performance impact when using \acp{MFE}. If it could be shown that a complex production facing web application could be transformed, to use a \ac{MFE} architecture \textit{without} a significant performance impact, the criticism regarding performance can be rejected. The problem can be condensed into the following research question:

\begin{center}
\begin{tabular}{p{0.28\textwidth}p{0.60\textwidth}}
     \textbf{Research Question:} & Is it possible to transform a monolithic complex production facing web application, to use a \ac{MFE} architecture, without \ac{UX} being significantly affected?
\end{tabular}
\end{center}

\section{Hypothesis}
It is possible to implement a monolithic complex web application using a \ac{MFE} architecture, without a large negative impact on performance and \ac{UX}.

\section{Purpose}
The purpose is to evaluate the feasibility of using \acp{MFE}, regarding \ac{UX}. The impact on \ac{DX} will not be evaluated. Assuming that \ac{DX} is positively impacted, there is a value in knowing if there is a trade-off between \ac{DX} and \ac{UX} or if \ac{UX} impacts are negligible.

\section{Goal}
The goal is to try to transform a monolithic complex web app, to using a \ac{MFE} architecture and compare performance between the monolithic web page and the \ac{MFE} web page. The attempt is that the \ac{MFE} page will have similar performance to the monolithic page, as this would mean that only \ac{DX} has to be considered when evaluating the use of \acp{MFE}.

\section{Tasks}
Initially a literature study will be conducted with two purposes:

\begin{enumerate}
    \item Evaluate the different methods for implementing a \ac{MFE} architecture.
    \item Decide on quantifiable performance metrics that have shown an impact on \ac{UX}. Likely metrics could be bundle size, time to first interaction, or time to first render. As there exists extensive research into this field, the chosen metrics will be a very trustworthy measurement of \ac{UX} impact.
\end{enumerate}

When an implementation and evaluation method is chosen, a complex monolithic web application will be re-implemented, using a \ac{MFE} architecture. When the modified web page is created the original and modified page can be compared, using the chosen performance metrics.

Finally the results will be evaluated and analyzed. The research question will be answered. All of this will be compiled into the thesis.

\section{Research Methodology}

\section{Old Method}
The project will use an empirical method. There exists quantifiable measurements, and the correlation between these and \ac{UX} are very extensively proven. Therefore, there exists a good foundation for conducting performance tests on web pages, and then from analysis of these measurements deduce \ac{UX} impact.

The biggest flaw with the chosen method is that it does not provide any method for proving that \acp{MFE} lead to a negative \ac{UX} impact. If the modified page is significantly worse than the original page it only proves that this specific page became worse, which could be because of a poor implementation. If the modified page has similar performance as the original page it proves that it is possible to create a \ac{MFE} web page with no significant performance degradation. Therefore, the hypothesis can be proven, but not rejected.

\section{Supervisor and Examiner}

My supervisor from KTH will be Martin Monperrus. My examiner from KTH will be Benoit Baudry.

The project will be conducted at DigitalRoute who will provide Tommy Gunnarsson as a supervisor. They will also provide me with any necessary equipment like a computer, and access to all of the development tools used at DigitalRoute.

\section{Eligibility and study planning}

All my courses from my bachelor are completed. I have completed more than 60 credits of advanced courses in my master. During my master thesis, I will conduct one 7.5 credit course, and after my master thesis I will have finished all courses for my master.

\section{Milestone chart}

The project will start on 13 January and end on 29 May. There will be the following milestones:

\textbf{7 February:} The project plan and literature study is finished. At this stage the method for implementation and performance evaluation will be chosen.

\textbf{17 April:} The modified web page will be finished. At this stage performance tests can be conducted.

\textbf{15 May:} The first draft of the Thesis will be finished. If it is accepted a thesis presentation date can be chosen.

\textbf{22 May:} The project presentation has been conducted and peer reviews have been provided.

\textbf{29 May:} The final thesis report is submitted.

\chapter{Background}

\section{Micro Frontends}
Micro frontends is an architectural style for scaling \ac{FE} development, so that many teams can work on the same project by enabling team independence \cite{Jackson2019}. It relies on dividing an \ac{FE} into multiple independently deliverable components that can be composed into a cohesive \ac{FE}. \acp{MFE} are related to microservices and share many of the same characteristics \cite[ch.~1]{Gears2020}. An important common aspect is the possibility for independent deployability \cite{Jackson2019}, where teams can deploy any changes to software owned by them, without affecting other teams. Some of the benefits from using \acp{MFE} are: simple decoupled codebases, independent deployment, autonomous teams \cite{Jackson2019}, and greater customer focus \cite[ch.~1]{Gears2020}.

According to some \acp{MFE} are related to vertical slicing \cite[ch.~1]{Gears2020}, a software decomposition strategy based on composing software in functionally coherent slices that fully implement features \cite{Ratner2011}. This is opposite to horizontal slicing, where software is composed of modules that fulfill the same technological purpose like database-, entity-, controller-, boundary-layer \cite{Ratner2011}.

There are many different integration approaches, for bundling the final \ac{FE} into a combined entity. The main categories are build-time integration, server-side integration, and client-side integration \cite{Jackson2019, Gears2020}. There is also route based integration or loosely coupled \acp{MFE} where the \ac{FE} consists of separate web pages served on different routes, that are connected with links \cites[ch.~2]{Gears2020}{ Yang2019}\cite{Yang2019}. Build-time integration does not fit all definitions of \acp{MFE} \cite{Gears2020} and there exists a consensus that using build-time integration looses many of the benefits of using \acp{MFE} \cite{Jackson2019}.

Server-side integration is when a \ac{FE} is split into fragments, which are combined in run-time \cite{Jackson2019, Gears2020}. This can be done by using Server Side Includes \cite{Gears2020,Jackson2019,Fagan}. A drawback with using server-side integration is that it does not enable dynamic client interactions. \cite[ch.~4]{Gears2020}. Dynamic in this case refers to the client application being able to update based on user interactions, without a page reload.

Client-side integration techniques are the most discussed \acp{MFE} techniques, as many modern web pages require dynamic application design. Client-side integration techniques can be categorised into many categories like: iframes, web components, and meta frameworks (also know as unified single page applications). \textbf{cite at least jackson and gears. Maybe also the google scholar one}

\section{Client-Side Integration Techniques}

% \section{\ac{MFE} frameworks}


\chapter{Related work}


\section{Micro Frontends}
Micro frontends is a front end technique that originates from microservices \cite{Jackson2019}. Where microservices aim to solve scalability problems problems in the back-end, micro frontends aim to solve the same problems in the front-end, by applying many of the same concepts and methods. There does not exist one single definition but one of the introducers of micro frontends, ThoughtWorks, define microfrontends as:
\blockquote{An architectural style where independently deliverable frontend applications are composed into a greater whole \cite{Jackson2019}}

An important common aspect is the possibility for independent deployability \cite{Jackson2019}, where teams can deploy any changes to software owned by them, without affecting other teams. The way this is achieved is by using vertical slicing \cite[ch.~1]{Geers2020}, a software decomposition strategy based on composing software in functionally coherent slices that fully implement features \cite{Ratner2011}. This is opposite to horizontal slicing, where software is composed of modules that fulfill the same technological purpose like database-, entity-, controller-, boundary-layer \cite{Ratner2011}. The evolution of decomposition strategies are presented in Figure \ref{fig:vertical-slicing}. Some of the promised benefits from using micro frontends are: simple decoupled codebases, independent deployment, autonomous teams \cite{Jackson2019}, and greater customer focus \cite[ch.~1]{Geers2020}. 

There are many different integration approaches for bundling the final front end into a combined entity. The main categories are: 1) build-time integration, 2) server-side integration, and 3) client-side integration \cite{Jackson2019, Geers2020}. There is also route based integration or loosely coupled micro frontends where the front end consists of separate web pages served on different routes, that are connected using only hyperlinks \cites[ch.~2]{Geers2020}{Yang2019}. Build-time integration does not fit all definitions of micro frontends \cite{Geers2020} and there exists a consensus that using build-time integration looses many of the benefits of using micro frontends \cite{Jackson2019}.

Server-side integration is when a front end is split into fragments, which are combined in run-time \cite{Jackson2019, Geers2020}. This can be done by using Server Side Includes \cite{Geers2020,Jackson2019,Fagan}. A drawback with using server-side integration is that it does not enable dynamic client interactions. \cite[ch.~4]{Geers2020}. Dynamic in this case refers to the client application being able to update the graphical user interface, based on user interactions, without a page reload.

Client-side integration techniques are the most discussed micro frontends techniques, as many modern web pages require dynamic application design. Client-side integration techniques can be categorised into categories like: iframes \cites{Jackson2019}[ch.~2]{Geers2020}, web components \cites{Jackson2019}[ch.~5]{Geers2020}, and meta frameworks (also know as unified single page applications) \cite[ch.~6]{Geers2020}.


\section{Client-Side Integration Techniques}

The most naive integration method is using iframes which is a web native standard \cites{Jackson2019}[ch.~2]{Geers2020}. Using iframes are in essence using a micro frontend architecture, and they were officially introduced with the HTML 4.01 standard in 1998 \cite{Raggett1999}. Iframes comes with some negative trade-offs like performance overhead, accessibility problems, search engine optimization problems, and the lack of layout control \cite[ch.~2]{Geers2020}.

A more modern alternative to iframes are web components that is a new collection of web native standards \cite{MDNWebDocs}. They include APIs that allow custom elements to be defined and shadow DOM which is an API for encapsulating elements \cite{MDNWebDocs}. Shadow DOM allows styling and other properties to be scoped to a specific part of a web page, to limit interference between different components \cite{WebComponents.org}. It is not necessary to use custom elements together with shadow DOM, and the appropriate combination can be applied depending on the use case \cite{Geers2020}.

Web components are a low level construct, and therefore many frameworks exist that uses web components as a compilation target. Polymer is a large framework suite, that can be used to create web components \cite{ThePolymerProject,ThePolymerProjecta}. Polymer is developed by Google \cite{ThePolymerProjectb} and used in Google products like Youtube and Google Earth \cite{ThePolymerProjectc}. SkateJS \cite{SkateJS}, Slim.js \cite{Slim.js,Slim.jsa}, hybrids \cite{Lubanski}, and snuggsi \cite{DevPunks} are other light weight web component frameworks. Stensil is a web component tool set that allows applications written in the traditional frameworks Angular, React, Vue, Ember to become web components \cite{StencilJS}. Angular \cite{Googlec} and Vue \cite{Vuejs} has native support for creating web components.

Web components are supported in all major browsers: Edge, Firefox, Safari, Chrome and Opera \cite{WebComponents.orga}. There are fewer drawback with using web components compared with using iframes, but some still exist. The largest issue is that web components don't support progressive enhancement, which is when the front end is prerendered on the server before being served \cite[ch.~5]{Geers2020}. This can become a problem for performance, accessibility and search engine optimization. Another drawback, compared with using compilation-time integration is the lack of static annotations and analysis. Web components can be internally statically analysed but there does not exist any mechanism to share static annotations across the boundary of a web component.

There exists micro frontend frameworks that do not utilize web components like react-async-component \cite{Matheson}. One of the most popular and extensive frameworks is single-spa \cite{Single-spa}. It is a shell application that includes other application written in other frameworks \cite{Single-spa}. This kind of framework is called a unified single-page app, as it wraps other single-page-applications into one cohesive application \cite{Geers2020}. Single-spa requires that the root application be written as a single-spa application, which implies a large migration cost to existing applications \cite{Single-spa}. No micro frontend frameworks provide static analysis between the application boundaries.


\chapter{Surveying the Industry}
\label{sec:interviews}

A qualitative survey was conducted with five industry experts, to understand how micro front-ends are used in practice and what problems they are applied to. The purpose was to discover aspects that can not be found in existing literature, and compile a common collection of micro front-end knowledge. This contributes to a knowledge base that can be used for developing Plutt and to help other micro front-end research.

\section{Intervied Protocol}

The experts interviewed were chosen based on their contributions to improving micro front-end methodologies or technologies, as well as having an extensive experience in working with micro front-ends. Some were found by their contributions being very public, and others were found recursively by asking the other experts.

To minimize the risk for bias, the questions were standardized. This also made it easier to compare the answers and understand what the experts agreed on and disagreed on. The questions were open ended, and unstructured counter questions were asked to further extract relevant knowledge. The interviews were between 60 and 90 minutes long\footnote{One interview took 4 hours, even though I had planned for 60-90 minutes. A lot of that time was discussing implementation details that is not relevant to the report. So the \textit{relevant} content is about the same as other interviews. Is that noteworthy, or should I just write that the interviews took 60-90 minutes?} and because of large geographical distances they were conducted over video call. They were recorded and can accessed by request.

\subsection{Interviewees}

\begin{description}
    \item[Michael Geers] is a developer who has worked many years for \textit{neuland - Büro für Informatik}. At neuland - Büro für Informatik he has helped large scale e-commerce web pages to use using micro front-ends. Michael is also the creator of \texttt{https://micro-frontends.org/} which at the time of writing is the first google result when searching for \textit{``micro frontends''}, and which is therefore a influential introduction to micro front-ends for many developers. He has been invited to hold talks about micro front-ends on many conferences and is the author of \textit{Micro Frontends in Action} \cite{Geers2020}.
    
    \item[Joel Denning] is a developer who was working on Amazon when he was tasked with solving the problem that their monolithic front-end was a development bottleneck. Inspired by Amazons famous microservice architecture, he developed a micro front-end architecture, which later became the foundation for single-spa. Single-spa is now one of the largest micro front-end composition technologies. He has also worked at Canopy as a front-end team lead until mid 2019, and have since started working as a consultant, helping companies with migrating to and using single-spa.
    
    \item[Luca Mezzalira] was mentioned by three of the other experts as being a central authority in the subject of micro front-ends. As a chief architect and VP of Architecture he has been a vital part of building the architecture for DAZN, a streaming platform with millions of concurrent global users, and hundreds of developers. He regularly speaks at conferences where he promotes the pragmatic micro front-end approach used by DAZN. He is also the author of \textit{Building Micro-Frontends} \cite{Mezzalira2021a}.
    
    \item[Zackary Jackson] has worked as a Senior Frontend Developer at Fiverr and Senior Frontend Architect at Starbucks. At both companies he has worked with refactoring and restructuring the monolithic front-ends into micro front-end applications. He has a background in providing web experiences to users with a slow internet connection, which explains his dedication to minimizing bundle sizes. His work on reducing code duplication lead to him becoming a maintainer of the famous JavaScript module bundler webpack. Zackary introduced the flagship feature for webpack version 5, module federation, which enables micro front-ends to share modules and components, with the primary goal of reducing bundle sizes fetched by client devices.
    
    \item[J\'er\'emy Colin] is a Senior Software Engineer on a Platform Team at \textit{Zalando}. Zalando is famous for their micro front-end suite mosaic, and is often mentioned for being on the forefront of introducing cutting edge micro front-end technologies. J\'er\'emy works with creating their next-generation web architecture, \textit{Interface Framework} \cite{Colin2018}, where parts are heavily inspired from relay \cite{FacebookInc.a}, which J\'er\'emy worked extensively with at \textit{Searchmetrics}. Interface Framework introduces cutting edge concepts like bundling data dependencies together with micro front-end fragments, a recommendation engine that dynamically selects fragments for the optimal user experience, and server side composed isomorphically rendered micro front-ends.
\end{description}

\subsection{Interview Questions}

The questions asked during the interviews where a mix of standardized questions and personalized questions that were related to the interviewees specific contributions and experience. Additional follow up questions were asked for a deeper understanding of the answers. The 12 standardized Interview Questions (IQ) are the following:

{
\setlist[enumerate,1]{label={\textbf{IQ \arabic*}}}
\begin{enumerate}
\item Could you tell me shortly about your experience with micro front-ends?

\item What is your definition of micro front-ends?

\item What would you say that the relation between microservices and micro front-ends is?

\item When should an organization look into using micro front-ends?

\item What are the benefits of micro front-ends?

\item Do you think that micro front-ends provide better UX?

\item How many micro front-ends should a team own?

\item How do you decide on where to place boundaries between micro front-ends?

\item What are your thoughts about team responsibilities when working on a micro front-end architecture? How and by whom is the orchestration handled?

\item Can the orchestration layer include user authentication and other global responsibilities?

\item Where is this field going in the future? What will you be working on?

\item Is there anything else you think I should know?
\end{enumerate}
}

The purpose of the first question was to fully understand the interviewees background, and to understand how that could effect the other questions. To understand what a micro front-end is, and to further understand the point of view of the interviewees answers, IQ 2 and 3 were asked. The later answers differed because the different interviewees had different definitions of micro front-ends, and where therefore discussing different aspects.

IQ 4-6 were meant to understand why micro front-ends are used, and what problems they aim to solve. IQ 7-10 were asked to understand how micro front-ends are and should be used. The last two questions aimed to complement the other questions with anything that was not discussed during the interview.

\textbf{Note: This section could be linked to research questions instead.}

\section{Survey Results}

\subsection{What is a micro front-end?}

The definition is very different when asking different experts, however there were three main aspects that were mentioned, even though not everyone agreed on them. They were independent deployability, isolation, and organizational alignment to business domain.

\textbf{Joel's} definition of micro front-ends is \blockquote{An independently deployable chunk, of your front-end web application.} Joel was humble to that the definition is not fully defined yet, and still evolving, but emphasizes that he believes that independent deployments is the most important aspect, and that build-time integration is a ``monolithic build'' and ``monolithic deployment''. He motivates this both by his definition of microservices, and the reason he thinks micro front-ends are being used:

\blockquote{The problems that micro front-ends solve, are largely organizational problems, not technical problems. And I think that one of the biggest organizational problems that is solved, is [...] different parts of an organization, being able to act independently. Being able to release their code to production without getting approval from everyone else.}

His rough definition of microservices is that microservices are only microservices if they are:

\begin{itemize}
    \item running on a seperate process
    \item communicating via network
    \item independently deployable
    \item have their own data store
    \item have their own build CI/CD process
\end{itemize}
It is impractical or even impossible for micro front-ends to communicate via network, and they are always executed on the same thread. The \ac{DOM}, which can be seen as a replacement for a data store, is shared, which means that they are accessing and mutating the same data store. The remaining technical aspects are independent deployability and separate build processes.

A controversial opinion is that Joel believes there can exist micro front-ends that are non visual. That a micro front-end can provide a service to other visual micro front-ends that are visual.

\textbf{Zackary's} definition is that a micro front-end has to be independently deployable and stand-alone. It was especially important that a micro front-end should work without any dependencies on it's environment, and this is an aspect that contradicts Joel's non visual micro front-ends. Non visual micro-frontends rely on other micro front-ends to update the \ac{DOM}, and if a micro front-end relies on another micro front-end, it is not self contained. Zackary was not against the concept and it's utility, but did not agree on that he would call it a micro front-end.

Zackary mentioned that the back-end is a much easier monolith to break apart into microservices, than the front-end is to break apart into micro front-ends. This relationship is based on that there exists many more technical constraints on the front-end than on the back-end.

\textbf{Michael and Luca's} point of view is much more an organizational. Michael is open to many different definitions but believe that micro front-ends work as a tool to align the back-end with the front-end, using fully vertically aligned teams. Luca does not agree that vertical alignment is necessary, and has a very concrete definition. It doesn't mention visual aspects in any way, and only focuses around organizational aspects. \blockquote{A micro-frontend represents a business domain that is autonomous, is independently deliverable, and is owned by a team.} This definition is very vague from a technological perspective, but is very strict in how they should be applied in an organisation. Both Michael and Luca use \textit{Domain Driven Design} to design their systems.

\textbf{For J\'er\'emy} micro front-ends are an isolated piece of UI. They are isolated in the sense of coupling, and not in the sense of deployments. The most important aspect for J\'er\'emy is that micro front-ends simplifies complexity management. This is very similar to Zackary's point of view, with the exception that J\'er\'emy allows synchronized deployments, and even mono-repositories in his definition.

To summarize, most agree on that microservices is a technical tool to solve an organizational problem. This is done by allowing teams to work more independently. Some believe the independence originates from the strictly decoupled design, while others believe it originates from independent deployments and mostly decoupled design. A micro front-end can possibly be a non-visual component that provides services to other micro front-ends, and that is tied to if the micro front-ends have to be strictly stand-alone or not. Micro front-ends borrow many aspects from microservices, but because of the vastly different execution environments and different constraints, they are also different in many ways. The area of microservice research can be seen as a source of desirable end-results and principles, and not a strict source for specific methods and processes.

\subsection{Why are micro front-ends being used?}

There was a clear consensus that micro front-ends solve complexity problems that arise when there are too many developers working on the same front-end. Solving the scalability problems, can have other indirect positive consequences, like a higher software quality, fewer published bugs, and counterintuitively higher performance. 

Joel was the only interviewee that used microfrontends primarily for ... less code duplication. Michael asset management css bla bla.

\textbf{Michael:} If a monolith is has grown to be too complicated. MFEs gives you a solution to give you clear boundaries. It also gives you a tool to gradually rewrite a large front-end. Like for migrating to a new technology.

For Michael it was always for organizational reasons. For creating vertically aligned teams. The benefit for him is that any time he wants to change or add a feature, he only has to discuss it and align it with one team that owns that feature domain. This is opposite to having to align multiple teams to add or change one feature. Like database, back-end and front-end teams.

A desirable goal is to have as much work and communication as possible being done inside teams, and reducing the need for communication and synchronization between teams.

\textbf{Luca:} Start using at 50 devs.

- Independent deployability

- Fault isolation

- Smaller and easier to work with

- Live support. Quick to do hotfixes.

\textbf{Zackary and Joel:} For a no compromise user experience. Both use micro front-ends to minimize initial bundle load, to maximize metrics like tti or time to first render


\subsection{How are micro front-ends used in practice?}

\textbf{Michael:} Pages are a good division to start with. DDD, and sets of pages are very easy to align with \textit{domains}, with some shared components sprinkled in.

From michael: ``\textbf{No cross-team API communication:} To do its work, a micro frontend should only talk to the backend infrastructure of its team. A micro frontend from Team A would never directly talk to an API endpoint from Team B. This would introduce coupling and inter-team dependencies. Even more important, you give up isolation. To run an test your system, the system from the other team needs to be present. An error in Team B would also affect fragments from Team A.''

\textbf{Remember to mention layout as a side effect somewhere}

\textbf{Mention no shared state. Like with micro services.}

Luca was very cautious in answering this, and referenced me to an orreiley talk where Mark Richards said ``Everything in software architecture is a trade-off.'' 

\section{(old. this is just notes that can be ignored) Interview with Michael Geers}
Michael Geers is a developer working for \textit{neuland - Büro für Informatik}. He has worked with large scale e-commerce projects using micro front-ends for 11 (not really) years. Michael Geers is also the creator of \texttt{https://micro-frontends.org/}, the author of \textit{Micro Frontends in Action} \cite{Geers2020}, and has been invited to hold talks about micro front-ends on many conferences.

An interview with Michael Geers was conducted, to ensure that any findings were based on the state of the art in micro front-ends. The interview was centered around Michael's practical experience with micro front-ends, the problems he has faced, and what solutions exists to solve those problems.

- He has not used unified single page applications, because the drawbacks do not fit his purpose

- Shell applications are very important to keep lean. If the integration layer should have any business logic like user authentication or x, it should have a dedicated team managing it.

- There are different ways of using micro front-ends, vertical slicing or separated back-end and front-end, using a microservice/micro front-end architecture.

- Portals is a cool new technology (https://wicg.github.io/portals/)

- Micro Front-ends does not inherently have any positive effects to the user experience of an app. It can however have indirect positive effects from increased software quality. Example: CSS from interview.

- DDD

\subsection{Asset management}
\chapter{Design of Plutt}

\section{Goals of Plutt}

Plutt aims to improve on other micro front-end solutions regarding responsibilities and ownership in decision-making in an organization. The primary decisions plutt targets are: 1) deciding when to update a micro front-end; 2) deciding where and how to host a micro front-end. As a secondary goal, plutt aims to achieve this with no boilerplate code.

The primary reasoning originates from Michal Geer's organization model of highly decoupled vertically aligned teams, and the concept of independent deployability mentioned by Joel Denning, Luca Mezzalira, and Zackary Jackson. A development team that creates a micro front-end should be able to deploy an update whenever it best suits them. There also has to exist a mechanism for how to handle breaking changes, as breaking changes impact other teams. Teams that depend on a micro front-end should be able to decide when they want to update a micro front-end to a newer breaking version, as it can require refactoring in their source code. Without a mechanism for providing lock-step deployment, the decision of when to update is wrongfully always up to the team providing a micro front-end.

Luca Mezzalira's definition of micro front-ends include autonomy, which can extend into team autonomy in choosing technologies and hosting solutions. Most micro front-end solutions require that the dependent of a micro front-end has to know where and how to fetch a micro front-end. If a micro front-end changes its hosting solution, it requires synchronized updates between two teams, which contradicts autonomy. This contradiction can be solved by rerouting requests from the old location to the new location, but a more robust solution would be to provide access transparency to dependents. In a type-safe environment, access transparent solutions have to provide static type information about a micro front-end component.

To fulfill the mentioned goals, plutt provides the following properties:

\begin{enumerate}
    \item Lock-step deployment using semantic versioning
    \item Technology heterogeneity
    \item Access transparency
    \item Static type information\footnote{Don't forget to mention this}
\end{enumerate}

\section{Architecture}

% overall architecture (how many components are there, what are the main responsibilities of each component, etc)

Plutt is a build tool that generates client-side integrated micro front-end applications. Plutt uses a react component as input and outputs two artifacts: a \textit{Plutt Application} (composed of the original Component and a \textit{Wrapper}) and a \textit{Proxy}. To serve plutt applications plutt comes with a \textit{Plutt Server}. Figure \ref{fig:plutt-architecture} shows a comparison between a monolithic application and an application that uses plutt. A component can recursively include other components or JavaScript modules.

A plutt application is composed of a wrapper and a component. The wrapper makes the component mountable, unmountable, and exposes methods to propagate and update properties. A proxy knows how to fetch and use the corresponding plutt application. The proxy is included in a parent application and propagates property values and updates. To achieve technology heterogeneity, plutt generates a different framework native proxy for every supported framework. All necessary information about how to fetch and use a plutt application is inserted into the proxy at compile time. This makes the proxies access transparent. The proxys API is the same as the corresponding components API, and the parent application does not have to configure the location of the plutt application. Neither the team who provides the parent application or the team who provides the plutt application has to implement any code to fetch, mount, unmount, or propagate properties.

\begin{figure}
    \centering
    \includegraphics[width=\linewidth]{images/plutt-architecture.pdf}
    \caption{Plutt Architecture compared to an equivalent monolithic front-end application.}
    \label{fig:plutt-architecture}
\end{figure}

To facilitate lock-step deployment, plutt applications can be stored on a plutt server that automatically upgrades requests to the latest available plutt application that is non breaking. That way minor updates will automatically be published whenever a plutt application is updated, while breaking updates are implemented when a parent application is refactored. This gives every team the correct responsibilities that facilitates independent deployments.


\section{Details}
To understand how plutt works there are four perspectives that should be considered: Developing a plutt application, plutt application at run time, serving plutt applications, and using a plutt application.

\subsection{Developing a Plutt Application}

Developing a plutt application is not very different from developing a component in a monolithic application. Plutt provides build capabilites through a \ac{CLI}. When building, plutt starts with extracting the project name, version, and the public path that the application will be located at (host path). Plutt extracts this from a JSON document in the project directory (\texttt{package.json}).

Secondly plutt generates the application. It does this by generating the wrapper, which in turn imports the source component, and exports functions that mount the source component to a provided \ac{DOM} node, unmount the source component, and update the source component. Plutt then uses webpack to transpile and bundle the project into one JavaScript file. The name of the bundle is based on the project name and version, so that plutt server knows what asset and what version of the asset, the file contains.

If the source component is implemented using TypeScript, type information is extracted at this step, and type definition files are created. The proxies should be API equivalent to the source component so the same type definitions can be used for the proxies. Plutt only supports type definitions for the proxy of the same type as the source component, as type information is differently structured for different frameworks. This could be extended with sophisticated analysis and conversion.

Lastly the different proxies are generated. The proxies are generated with the host location of the plutt app hard coded. The location is a concatenation between the configured host path and the derived file name.

The proxy can be uploaded to a package registry that can be included in a parent application, and the plutt application has to be uploaded to a hosting location that is available by the end user, and the same location as previously configured.

\subsection{Plutt Application at Run Time}



\subsection{Serving Plutt Applications}

A plutt application can be uploaded to a static file server. If the same version and name is used every time, the plutt application could be updated by replacing the plutt application file on the file server. In this case the developers of the plutt application is the team that decides when the user facing web page is updated, which could introduce problems if an update introduces breaking changes.

Another option is to keep all versions of a plutt application on a static file server, and use the unique file names they are generated with. This is done by updating the configured version every time a new version of a plutt app is created. As every version of the proxies only points to a unique plutt application, new versions of the plutt application will only be user facing when the developers of the parent application decides to update their proxy.

An optimal solution would be that the team providing a micro front-end decides when non breaking updates are published into production, and the parent application team decides when to update to newer breaking versions. In other words the best solution would be to provide both independent deployments, and API safety using lock-step deployment. Plutt server is a plutt application repository that provides version safe independent deployments.

Whenever plutt server receives a request, the server examines if there is a higher version of the requested asset that is non breaking. If there exists a more recent non breaking version, plutt server will upgrade the request to the highest available non breaking version. This means that the users of a plutt application will always get at least the version that existed when they included the proxy in there application, but could be provided with updates as long as they are non breaking. The providers of a plutt application own the responsibility of publishing non breaking updates. Whenever a breaking update is introduced, the teams can use lock-step deployment to gradually introduce the new update. For the team that uses a plutt application, this means updating their imported proxy.

\subsection{Using a Plutt Application}


\section{Implementation}

answer to RQ1


\chapter{Old Design of Plutt}

\textbf{Write something short here}

\section{Goals of Plutt}

Plutt aims to improve on other micro front-end solutions regarding responsibilities and ownership in decision-making in an organization. The primary decisions plutt targets are: 1) deciding when to update a micro front-end; 2) deciding where and how to host a micro front-end. As a secondary goal, plutt aims to achieve this with no boilerplate code.

The primary reasoning originates from Michal Geer's organization model of highly decoupled vertically aligned teams, and the concept of independent deployability mentioned by Joel Denning, Luca Mezzalira, and Zackary Jackson. A development team that creates a micro front-end should be able to deploy an update whenever it best suits them. There also has to exist a mechanism for how to handle breaking changes, as breaking changes impact other teams. Teams that depend on a micro front-end should be able to decide when they want to update a micro front-end to a newer breaking version, as it can require refactoring in their source code. Without a mechanism for providing lock-step deployment, the decision of when to update is wrongfully always up to the team providing a micro front-end.

Luca Mezzalira's definition of micro front-ends include autonomy, which can extend into team autonomy in choosing technologies and hosting solutions. Most micro front-end solutions require that the dependent of a micro front-end has to know where and how to fetch a micro front-end. If a micro front-end changes its hosting solution, it requires synchronized updates between two teams, which contradicts autonomy. This contradiction can be solved by rerouting requests from the old location to the new location, but a more robust solution would be to provide access transparency to dependents. In a type-safe environment, access transparent solutions have to provide static type information about a micro front-end component.

To summarize, plutt could fulfill its goals by providing the following qualities:

\begin{enumerate}
    \item Lock-step deployment using semantic versioning
    \item Technology heterogeneity
    \item Access transparency
    \item Static type information\footnote{Don't forget to mention this}
\end{enumerate}

\section{Architecture}

Plutt is a build tool that generates client-side integrated micro front-end applications. The input is a react component and plutt outputs two artifacts\footnote{Is artifact the right word?}: a \textit{Plutt Application}, containing a \textit{Wrapper}, and a \textit{Proxy}. To serve plutt applications plutt comes with a \textit{Plutt Server}. Figure \ref{fig:plutt-architecture} shows a comparison between a monolithic application and an application that uses plutt. A component can recursively include other components or JavaScript modules.


\begin{figure}
    \centering
    \includegraphics[width=\linewidth]{images/plutt-architecture.pdf}
    \caption{Plutt Architecture compared to an equivalent monolithic front-end application.}
    \label{fig:plutt-architecture}
\end{figure}

\subsection{Plutt Application}
The input component is wrapped in a wrapper that plutt generates. The wrapper is a JavaScript module that exposes functions to mount and unmount the component on an HTML element. The wrapper also propagates properties to the component and exposes a function to update the properties. The wrapper and component is compiled into one bundle by plutt, which is the plutt application. Plutt only supports components to be react components, but can be extended to support components implemented using other frameworks.

This bundle can be fetched in run-time using dynamic imports in JavaScript, like Joel Denning mentioned\footnote{Mention dynamic imports and esmodules}, and mounted on an HTML element. If a parent application fetches a plutt application at run time, the plutt application can be replaced in run time, which enables independent deployments.


\subsection{Proxy}
When compiling a plutt application, plutt compiles proxies that knows how to fetch, mount and unmount a plutt application. The proxies can also receive properties and property updates which it propagates to the mounted plutt application. To accomplish this plutt has to know where the plutt application will be hosted at compilation-time, which is inserted into the proxies.

The proxies are generated to be included in the parent application, which is why there are multiple different proxies. There is one proxy for every supported framework, which at this point is React and Vue. The proxy is a framework-native component which can be used exactly as if it was the original component. Code-wise it looks exactly the same as if the parent application would include the original component, given that the component is implemented using the same framework. The proxy provides access transparency and technology heterogeneity.

As the proxy does not contain business logic it works with any non-breaking version of the corresponding plutt application. Independent deployments can be achieved by replacing only the plutt application on the hosted location. The proxy does not have to be updated in the parent application at the same time.

\subsection{Plutt Server}
The plutt application could be hosted on a static file server, but that does not provide lock-step deployment. Every time a plutt application is replaced, the corresponding proxy would fetch the latest version, even if the latest version contains a breaking update. This gives the full ownership of updating to the team who is responsible for the plutt app, which could lead to the team developing the parent application to be forced to update their application at a time that does not suit them.

To solve this plutt appends the version to the filename of the plutt application, every time plutt compiles a plutt application. Every version of the plutt application should result in a unique file name, provided semantic versioning is used. Instead of replacing the plutt application at the host location, every new file can be appended to a repository with a unique hosting location. The parent application will get the version that was compiled with the proxy that is included. Table \ref{table:plutt-configuration} shows an example configuration and Table \ref{table:plutt-configuration-result} shows what values would be derived.

\begin{table}
\centering
\caption{Configuration values inputed into plutt}
\label{table:plutt-configuration}
\begin{tabular}{|l|l|}
\hline
name      & banner                           \\ \hline
version   & 2.1.1                            \\ \hline
host path & https://banners.com/plutt/banner \\ \hline
\end{tabular}
\end{table}

\begin{table}
\centering
\caption{The resulting values derived from the configuration values in Table \ref{table:plutt-configuration}. File name is the file name of the plutt application bundle, and full host path is the path that the proxy will fetch from.}
\label{table:plutt-configuration-result}
\begin{tabular}{|l|l|}
\hline
file name      & banner.v2.1.1.js                                  \\ \hline
full host path & https://banners.com/plutt/banner/banner.v2.1.1.js \\ \hline
\end{tabular}
\end{table}

The full ownership of updating the plutt application is moved to the team that develops the parent application. Using this method, the parent application team will not get non breaking updates automatically, which is also suboptimal. The optimal solution would be that the team providing a micro front-end decides when non breaking updates are published into production, and the parent application team decides when to update to newer breaking versions. To solve this issue, plutt comes with a server for serving plutt applications in a semantically safe manner.

Whenever plutt server receives a request, the server examines if there is a higher version of the requested asset that is non breaking. If there exists a more recent non breaking version, plutt server will upgrade the request to the highest available non breaking version.

This means that the users of a plutt application will always get at least the version that existed when they included the proxy in there application, but could be provided with updates as long as they are non breaking. The providers of a plutt application own the responsibility of publishing non breaking updates. Whenever a breaking update is introduced, the teams can use lock-step deployment to gradually introduce the new update. For the team that uses a plutt application, this means updating their imported proxy.

\subsection{Type safety}



\subsection{Summary or something}

\textbf{Write a short summary here}

For sake of open-science, the code is made publicly available at \textbf{insert link here}
\chapter{Evaluation}

singletons in general.

\section{DR}
Redux and Context. Global values and state. Compare bounded scope (esmodules) to unbounded scope (appended script tag).

\section{Real World Example}
Soft navigation links.
\include{sections/7-discussion}
\include{sections/8-conclusion}

% \chapter{Misc}

% \section{How to load the JS}
% Write about the decision of how to load/fetch JS.

% \begin{itemize}
%     \item Append script to page. 
    
%     I got the idea from this: (https://www.danielcrabtree.com/blog/25/gotchas-with-dynamically-adding-script-tags-to-html)
    
%     I used this: (https://stackoverflow.com/a/27468484)
%     \item Dynamic import. "Dynamic import is useful in situations where you wish to load a module conditionally, or on-demand." \cite{MDNWebDocs2020}
% \end{itemize}

% \section{Webpack vs Rollup}

\printbibliography

\end{document}
