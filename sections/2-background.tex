\chapter{Background}

\section{Microservices}
Microservices is a software development method where systems are divided into smaller parts, called microservices, that can be individually deployed \cite{Richardson2019}. In turn, this enables low coupling, high cohesion, and strong composibility \cites[ch.~1]{Newman2015a}{Richardson2019}. Microservices are focused on performing single tasks really well, and implementing functionality by composing multiple microservices. It also enables multiple developers to work on a common codebase while minimizing obstruction, which becomes more notable in larger systems, with a greater number of developers \cite{Newman2015a}. This is also referred to as independence.

A central aspect of microservices is the concept of ``vertical slicing'' as a decomposition strategy, which is an important facilitator of team independence \cite{Familiar2015,Ratner2011}. \citeauthor{Familiar2015} describes this as: 
\blockquote{Traditionally, we have used separation of concerns, a design principle for separating implementation into distinct layers in order to define horizontal seams in our application architecture. Microservice architecture applies separation of concern to identify vertical seams that define their isolation and autonomous nature. \cite[p.~xx]{Familiar2015}}


\section{Maybe write about self contained systems (SCS)}
http://scs-architecture.org/