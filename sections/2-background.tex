\chapter{Background}

\section{Micro Frontends}
Micro frontends is an architectural style for scaling \ac{FE} development, so that many teams can work on the same project by enabling team independence \cite{Jackson2019}. It relies on dividing an \ac{FE} into multiple independently deliverable components that can be composed into a cohesive \ac{FE}. \acp{MFE} are related to microservices and share many of the same characteristics \cite[ch.~1]{Gears2020}. An important common aspect is the possibility for independent deployability \cite{Jackson2019}, where teams can deploy any changes to software owned by them, without affecting other teams. Some of the benefits from using \acp{MFE} are: simple decoupled codebases, independent deployment, autonomous teams \cite{Jackson2019}, and greater customer focus \cite[ch.~1]{Gears2020}.

According to some \acp{MFE} are related to vertical slicing \cite[ch.~1]{Gears2020}, a software decomposition strategy based on composing software in functionally coherent slices that fully implement features \cite{Ratner2011}. This is opposite to horizontal slicing, where software is composed of modules that fulfill the same technological purpose like database-, entity-, controller-, boundary-layer \cite{Ratner2011}.

There are many different integration approaches, for bundling the final \ac{FE} into a combined entity. The main categories are build-time integration, server-side integration, and client-side integration \cite{Jackson2019, Gears2020}. There is also route based integration or loosely coupled \acp{MFE} where the \ac{FE} consists of separate web pages served on different routes, that are connected with links \cites[ch.~2]{Gears2020}{ Yang2019}\cite{Yang2019}. Build-time integration does not fit all definitions of \acp{MFE} \cite{Gears2020} and there exists a consensus that using build-time integration looses many of the benefits of using \acp{MFE} \cite{Jackson2019}.

Server-side integration is when a \ac{FE} is split into fragments, which are combined in run-time \cite{Jackson2019, Gears2020}. This can be done by using Server Side Includes \cite{Gears2020,Jackson2019,Fagan}. A drawback with using server-side integration is that it does not enable dynamic client interactions. \cite[ch.~4]{Gears2020}. Dynamic in this case refers to the client application being able to update based on user interactions, without a page reload.

Client-side integration techniques are the most discussed \acp{MFE} techniques, as many modern web pages require dynamic application design. Client-side integration techniques can be categorised into many categories like: iframes, web components, and meta frameworks (also know as unified single page applications). \textbf{cite at least jackson and gears. Maybe also the google scholar one}

\section{Client-Side Integration Techniques}

% \section{\ac{MFE} frameworks}

